\subsection{Motion}

The primary program functions for the movement of the Epson arm were dclearly defined, with each function performing a single command. Commands were chained together within Matlab. Every function was designed with a reversal function. Any error that breaks the program flow can only by undone by the reversal function, so this was absoulutely essential (especially during testing of the early stages of the overall program). The alternatives was to manually move the Epson arm, joint by joint, to a position from which a movement could occur. This was incredibly time consuming, so pre-built functions which would do this for us were essential.

\begin{itemize}
	
	\item Go Home
		\begin{itemize}
		\item This function moves the arm to a predefined home position. Care need to be taken when using this function, as it is not always possible for the arm to return to the home position, mainly due to degeneracy or self-obstruction. As such, this function is only ever called after calling the reversal of the previous function to run, if it had one.
		\item This is the default reversal function
		\end{itemize}
	\item Hide Epson Arm
		\begin{itemize}
		\item This moves the epson arm out from the view of the camera mounted above the system, ensuring that the camera has a clear, unobstructed view of the table. 
		\item The reversal function is Unhide Epson Arm
		\end{itemize}
	\item Unhide Epson arm
		\begin{itemize}
		\item This is the reversal of Hide Epson Arm. It returns the Epson arm to the home position. Simply calling Go Home would result in a controller failure due to degeneracy, so a custom function was necessary
		\end{itemize}
	\item Position Tool
		\begin{itemize}
		\item Given a a centroid location of a card, this function moves the end effector to the \textit{x-y} coordinates of the centroid.
		\item The reversal of this function was Go Home, as it does not make any unsual movements which could cause degeneracy of obstruction.
		\end{itemize}
	\item Set Tool Height
		\begin{itemize}
		\item This function controls the \textit{z} coordinates of the end effector. It uses predefined measurements of the height of the cards, so that the function can take an integer number defining how many cards high we want to be. For exampl, to pick up a card we go to height 0, which would actually be equal to half the height of the card, enabling the grippers to effectively pick it up. Height 1 would place the grippers in such a position where it could be a card stacked on top of another, and so on.
		\item This function is its own reversal, simply returning to a default operating height that avoids collisions with the cards and feducials.
		\end{itemize}
	\item Set Tool Angle
		\begin{itemize}
		\item Given the angle to the horizontal that the card is on (as determined by the vision processing), this turns the grippers to that angle. Great care was taken to ensure that any twisting motion was limited to be between 0 and 180 degrees. Turning further than 180 degrees, or into the negative range, ran the risk of damaging pneumatic tubing that powered the grippers. This range was chosen as there is still a siginificant amount of motion left before the risk of damage occured, and the range also ensured that the a card could always be picked up.
		\item This function is its own reversal
		\end{itemize}
	\item Display Card
		\begin{itemize}
		\item Shows the card to the audience - raises the grippers by 90 degrees.
		\item Is reversed by UnDisplay
		\end{itemize}
%	IMAGINATIVE NAMES FTW!
	\item UnDisplay Card
		\begin{itemize}
		\item Returns the grippers to their default vertical position.
		\end{itemize}
	\item Peek
		\begin{itemize}
		\item Turns a card up so that it can be seen by the camera. This places the card in a fixed, known position, allowing identification by the camera simpler.
		\item The reversal is UnPeek. 
		\end{itemize}
	\item Unpeek
		\begin{itemize}
		\item The reversal for Peek
		\end{itemize}
	\item Open Grippers
	\item Close Grippers
	
\end{itemize}