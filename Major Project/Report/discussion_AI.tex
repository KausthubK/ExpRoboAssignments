\subsection{AI}
One of the major unforseen issues that the program flow faced was that it would consistently compare a new card with itself and consider itself a match, leading to every card being put on the tower in the order that they were picked up. The issue was resolved by mainly that instead of having two states for cards (seen and unseen) the system needed to consider the newly read card differently so that it would only compare it with OTHER cards. Other than this the comparison functionality seemed to work consistently after that fix was introduced. It is also interesting to note that the split of motion functions into "peek" and "unpeek" came with its own share of problems while testing because when the program crashed from a failure during testing and development it would need to be manually pulled back into its working zone (similar degeneracy issues to hiding and unhiding the arm). The change that managed this was that at every error handling situation we would have to check if it was hiding or peeking and reverse that action before exiting the program with the error message. Once this change was made (prior to the presentation and demonstration) it performed without having such issues because by that point the errors in other areas of the programming (such as not being able to detect some cards) had been dealt with. The Hardware input dependent functions (such as survey field and read card and peek card) had to have these checks in place as majority of the error would be found in the physical end of the project. One of these issues that couldn't be solved was that the entire manipulator arm is a an open loop without any feedback for whether it was in the right position or whether it had picked up a card properly. In order to fix this the best way would be to apply touch sensors or flex sensors to the end effector that would allow us to deal with error situations when there is a malfunction such as not picking up a card. This could then be incorporated into the software testing procedures undertaken to ensure that the system was robust.