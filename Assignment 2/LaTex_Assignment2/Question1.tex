\newpage
\section{Question 1}
	
	
	d. Add your function to one of the above scripts and display the resulting lines as part of the laserShow script.  Include a few sample plots in your write up showing how your system performs. 
	
	
	e. Finally, corners can be characterised by the intersection of two lines that are nearly perpendicular.  Select two scans that contain corners (you may select them manually by stepping through the scans and identifying scans with a few corners) and find adjacent lines that are approximately perpendicular. (HINT: the dot product of perpendicular vectors describing two lines is approximately zero) 
	\subsection{Perpendicular Distance Function}	%a
		\lstinputlisting{./code/Q1/perpdist.m}
		\pagebreak
	\subsection{Least Square Minimization Algorithm}	%b
		\lstinputlisting{./code/Q1/leastsqmin.m}
		\pagebreak
	\subsection{Line Segmentation Algorithm}	%c
		\lstinputlisting{./code/Q1/lineseg.m}
		\linebreak
		\newline
		\subsubsection{Test Fitting}
			The following images are saved plots of sample data plots of increasing complexity where the original data is plotted with a red 'x' and the blue line through it is the approximated line segments. The relevant test scripts can be found in Appendix A [1].
			It's interesting to note that the sideways v test faces some inaccuracy in the approximation due the restriction on not allowing the index end points to be considered "corners". This is one of the shortcomings of using such a simple method of resolving the "broken data set" situation. A method that minimises that error as well would be much more difficult to program and much more intricate to specific case dealing.
			\begin{figure}[position = here]
				\begin{centering}
					\includegraphics[scale=0.5]{q1_test00}\\
					\caption[\textit{RPYAxes}]{Horizontal Line}
				\end{centering}
			\end{figure}
			\newline
			
			\begin{figure}[position = here]
				\begin{centering}
					\includegraphics[scale=0.5]{q1_test01}\\
					\caption[\textit{RPYAxes}]{Vertical Line}
				\end{centering}
			\end{figure}
			\newline
			
			\begin{figure}[position = here]
				\begin{centering}
					\includegraphics[scale=0.5]{q1_test02}\\
					\caption[\textit{RPYAxes}]{Slope (Through Origin)}
				\end{centering}
			\end{figure}
			\newline
			
			\begin{figure}[position = here]
				\begin{centering}
					\includegraphics[scale=0.5]{q1_test03}\\
					\caption[\textit{RPYAxes}]{Offset Slope}
				\end{centering}
			\end{figure}
			\newline
			
			\begin{figure}[position = here]
				\begin{centering}
					\includegraphics[scale=0.5]{q1_test04}\\
					\caption[\textit{RPYAxes}]{Inverted V}
				\end{centering}
			\end{figure}
			\newline			
			
			\begin{figure}[position = here]
				\begin{centering}
					\includegraphics[scale=0.5]{q1_test05}\\
					\caption[\textit{RPYAxes}]{Sideways V}
				\end{centering}
			\end{figure}
			\newline			
			
			\begin{figure}[position = here]
				\begin{centering}
					\includegraphics[scale=0.5]{q1_test06}\\
					\caption[\textit{RPYAxes}]{Perpendicular ZigZag)}
				\end{centering}
			\end{figure}
			\newline			
			
			\begin{figure}[position = here]
				\begin{centering}
					\includegraphics[scale=0.5]{q1_test07}\\
					\caption[\textit{RPYAxes}]{Broken Zig Zag}
				\end{centering}
			\end{figure}
			\newline			
			
			\begin{figure}[position = here]
				\begin{centering}
					\includegraphics[scale=0.5]{q1_test08}\\
					\caption[\textit{RPYAxes}]{Perpendicular Zig Zag (Sideways)}
				\end{centering}
			\end{figure}
			\newline			
			
			\begin{figure}[position = here]
				\begin{centering}
					\includegraphics[scale=0.5]{q1_test09}\\
					\caption[\textit{RPYAxes}]{Non-Perpendicular ZigZag}
				\end{centering}
			\end{figure}
			\newline			
			
			\pagebreak
		\subsubsection{Test 06 Step by Step}
		The following images take a step by step plotted walk through on each estimation made on a simple case such as the Zig Zag pattern shown in test 06. The important thing to note is the recursive method utilised in this code which splits this into a logical problem solving technique (similar to mergesort or quicksort algorithms) by selecting pivot-points 
			\begin{figure}[position = here]
				\begin{centering}
					\includegraphics[scale=0.5]{q106_0}\\
					\caption[\textit{RPYAxes}]{Test 06 Stepped Through}
				\end{centering}
			\end{figure}
			\newline			
			
			\begin{figure}[position = here]
				\begin{centering}
					\includegraphics[scale=0.25]{q106_1}\\
					\caption[\textit{RPYAxes}]{Test 06 Stepped Through}
				\end{centering}
			\end{figure}
			\newline						

			\begin{figure}[position = here]
				\begin{centering}
					\includegraphics[scale=0.25]{q106_2}\\
					\caption[\textit{RPYAxes}]{Test 06 Stepped Through}
				\end{centering}
			\end{figure}
			\newline			
		
			\begin{figure}[position = here]
				\begin{centering}
					\includegraphics[scale=0.25]{q106_3}\\
					\caption[\textit{RPYAxes}]{Test 06 Stepped Through}
				\end{centering}
			\end{figure}
			\newline			
			
			\begin{figure}[position = here]
				\begin{centering}
					\includegraphics[scale=0.25]{q106_4}\\
					\caption[\textit{RPYAxes}]{Test 06 Stepped Through}
				\end{centering}
			\end{figure}
			\newline			
			
			\begin{figure}[position = here]
				\begin{centering}
					\includegraphics[scale=0.25]{q106_5}\\
					\caption[\textit{RPYAxes}]{Test 06 Stepped Through}
				\end{centering}
			\end{figure}
			\newline			
			
			\begin{figure}[position = here]
				\begin{centering}
					\includegraphics[scale=0.25]{q106_6}\\
					\caption[\textit{RPYAxes}]Test 06 Stepped Through}
				\end{centering}
			\end{figure}
			\newline								
			
			\begin{figure}[position = here]
				\begin{centering}
					\includegraphics[scale=0.25]{q106_7}\\
					\caption[\textit{RPYAxes}]{Test 06 Stepped Through}
				\end{centering}
			\end{figure}
			\newline										

	\pagebreak
	\subsection{Laser Show ACFR Line Fitting}	%d
			\lstinputlisting{./code/Q1/laserShowACFR.m}
			\linebreak
			\begin{figure}[position = here]
				\begin{centering}
					\includegraphics[scale=0.2]{q1d_solution}\\
					\caption[\textit{RPYAxes}]{LaserShowACFR approximate}
				\end{centering}
			\end{figure}
			\newline	
			
	\pagebreak
	\subsection{Corner Detection}	%e
		\lstinputlisting{./code/Q1/findcorners.m}
		\linebreak
		
		\begin{figure}[position = here]
			\begin{centering}
				\includegraphics[scale=0.5]{q1_test10}\\
				\caption[\textit{RPYAxes}]{Non-Perpendicular Zig Zag Corner Test}
			\end{centering}
		\end{figure}
		\newline
		
		\begin{figure}[position = here]
			\begin{centering}
				\includegraphics[scale=0.5]{q1_test11}\\
				\caption[\textit{RPYAxes}]{Perpendicular Zig Zag Corner Test}
			\end{centering}
		\end{figure}
		\newline		
		\pagebreak
		\lstinputlisting{./code/Q1/test/Q1test_results/test_results.txt}
		\linebreak
		
		As you can see the corners listed match perfectly with those graphed on Fig 21 (Test 11).
		
		\lstinputlisting{./code/Q1/q1e.txt}
		\linebreak
		The above corner values relate to the plot of the line estimates of the Laser Show ACFR scan in figure 19.
		
		\subsection*{Test Code Listing}
			See Appendix A [9.1]